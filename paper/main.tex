\documentclass{article}
\usepackage{arxiv}

\usepackage[utf8]{inputenc}
\usepackage[english, russian]{babel}
\usepackage[T2A, T1]{fontenc}
\usepackage{url}
\usepackage{booktabs}
\usepackage{amsfonts}
\usepackage{nicefrac}
\usepackage{microtype}
\usepackage{lipsum}
\usepackage{graphicx}
\usepackage{natbib}
\usepackage{doi}



\title{Optimal Gradient Methods with Relative Inexactness}

\author{
	Рубцов Денис \\
	\texttt{rubtsov.dn@phystech.edu} \\
	%% examples of more authors
	\And
	Корнилов Никита \\
	\texttt{kornilov.nm@phystech.edu} \\
	%% \AND
	%% Coauthor \\
	%% Affiliation \\
	%% Address \\
	%% \texttt{email} \\
	%% \And
	%% Coauthor \\
	%% Affiliation \\
	%% Address \\
	%% \texttt{email} \\
	%% \And
	%% Coauthor \\
	%% Affiliation \\
	%% Address \\
	%% \texttt{email} \\
}
\date{}

\renewcommand{\shorttitle}{Optimal Gradient Methods with Relative Inexactness}
\renewcommand{\undertitle}{}
%%% Add PDF metadata to help others organize their library
%%% Once the PDF is generated, you can check the metadata with
%%% $ pdfinfo template.pdf
\hypersetup{
pdftitle={Optimal Gradient Methods with Relative Inexactness},
pdfsubject={Методы оптимизации первого порядка},
pdfauthor={Рубцов Д.Н., Корнилов Н.М.},
pdfkeywords={методы первого порядка, ускоренные методы, неточный градиент, относительный шум, Performance Estimation Problem},
}

\begin{document}
\maketitle

\begin{abstract}
Работа посвящена ускоренным методам гладкой выпуклой оптимизации первого порядка с градиентами, известными лишь с некоторой относительной погрешностью. Проведен обзор полученных ранее теоретических результатов об оценках максимально допустимой погрешности, сохраняющей линейную сходимость методов. С помощью анализа численного решения эквивалентной задачи полуопределенного программирования (техника PEP) показана достижимость этих оценок.

\end{abstract}


\keywords{методы первого порядка, ускоренные методы, неточный градиент, относительный шум, Performance Estimation Problem}

\section{Введение}

\bibliographystyle{unsrtnat}
%\bibliography{references}

\end{document}
